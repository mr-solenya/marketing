\documentclass{report}
\usepackage[extreme]{savetrees}
\usepackage{wrapfig}
\usepackage{graphicx}
\graphicspath{{./images/}}
\usepackage{amsthm}
\usepackage{url}
\usepackage{subfig}
\usepackage{lscape}
\usepackage{amsmath}
\usepackage{tikz}
\usepackage{mathdots}
\usepackage{yhmath}
\usepackage{cancel}
\usepackage{color}
\usepackage{siunitx}
\usepackage{array}
\usepackage{multirow}
\usepackage{amssymb}
\usepackage{gensymb}
\usepackage{tabularx}
\usepackage{booktabs}
\usetikzlibrary{fadings}
\title{Marketing Training Report}
\author{Tuneer Bhattacherjee\\Employee Id: 514370\\BhattacherjeeT@indianoil.in}
\makeindex
\begin{document}
	\maketitle
	\pagebreak
	\tableofcontents
	\pagebreak
	\chapter{Sanand Bottling Plant - Day 1}
	\section{Salient Points}
	\begin{itemize}
		\item Due to low land prices and government push, this particular bottling plant has much more space than what is required under the OISD guidelines.
		\item In addition to that, there is a 66 acre buffer area which is not required anymore according to the latest OISD guidelines so, the plant "occupier" or location in charge Shri. Joydev Ojha, DGM(P) has decided to utilize it to benefit the corporation in the following ways:-
		\begin{itemize}
			\item A 8MW solar plant was established in the buffer area which generates about Rs.3 lacs of electricity per day, part of which is used up by the factory and part is distributed to other IOCL facilities via the grid. It is important to note that according to the some regulations in the Gujrat solar power consumption policy, IOCL at max can only generate 50\% of their net electricity demand if they wish to stay on grid and share their power with other IOCL facilities using the same. This facility covers 66 acre of the buffer area. 
			
			\item  A 2acre lube storage facility (CFA). It is important to note here that lube being a high flashpoint product is an ``excluded product''. Therefore, storing it in buffer areas donot raise any safety concerns.
			
			\item 
		\end{itemize}
	\end{itemize}
	
	\section{Glossary and Abbreviations}
	
	\section{FAQs}
	
\end{document}