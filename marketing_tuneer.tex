\documentclass{report}
\usepackage[extreme]{savetrees}
\usepackage{wrapfig}
\usepackage{graphicx}
\graphicspath{{./images/}}
\usepackage{amsthm}
\usepackage{url}
\usepackage{subfig}
\usepackage{lscape}
\usepackage{amsmath}
\usepackage{tikz}
\usepackage{mathdots}
\usepackage{yhmath}
\usepackage{cancel}
\usepackage{color}
\usepackage{siunitx}
\usepackage{array}
\usepackage{multirow}
\usepackage{amssymb}
\usepackage{gensymb}
\usepackage{tabularx}
\usepackage{booktabs}
\usetikzlibrary{fadings}
\title{Marketing Training Report}
\author{Tuneer Bhattacherjee\\Employee Id: 514370\\BhattacherjeeT@indianoil.in}
\makeindex
\begin{document}
	\maketitle
	\pagebreak
	\tableofcontents
	\pagebreak
	\chapter{Sanand Bottling Plant - Day 1}
	\section{Glossary and Abbreviations}
	\begin{itemize}
		\item OISD = Oil Industry and Safety Directorate
		\item TT= Tanker Trucks
		\item Cavitation = 
		\item CFM= Cubic Feet per Minute
		\item SO= State Office
		\item HO= Head Office
	\end{itemize}
	\section{Introduction to Sanand LPG Bottling Plant - Salient Points}
	\begin{itemize}
		\item Due to low land prices and government push, this particular bottling plant has much more space than what is required under the OISD guidelines.
		\item In addition to that, there is a 66 acre buffer area which is not required anymore according to the latest OISD guidelines so, the plant "occupier" or location in charge Shri. Joydev Ojha, DGM(P) has decided to utilize it to benefit the corporation in the following ways:-
		\begin{itemize}
			\item A 8MW solar plant was established in the buffer area which generates about Rs.3 lacs of electricity per day, part of which is used up by the factory and part is distributed to other IOCL facilities via the grid. It is important to note that according to the some regulations in the Gujrat solar power consumption policy, IOCL at max can only generate 50\% of their net electricity demand if they wish to stay on grid and share their power with other IOCL facilities using the same. This facility covers 66 acre of the buffer area. 
			
			\item  A 2 acre lube storage facility (CFA). It is important to note here that lube being a high flashpoint product is an ``excluded product''. Therefore, storing it in buffer areas donot raise any safety concerns.
			
			\item 4 acres are being delegated to the pipelines division to facilitate the Kandla-Gorakhpur pipeline.
			
			
		\end{itemize}
		\item Product is sourced into the bottling plant using approximately 100 LPG TTs of 18-20 MTs from the following sources -
		\begin{itemize}
			\item Kandla port
			\item Pipawa port
			\item Varoda refinery
			\item Reliance refinery, Jamanagar
			\item Essar refinery, Jamnagar etc.
		\end{itemize}
	
		\item There are 8 TLDs which takes about 3-4 hours to completely decant all the trucks and this happens in about 4 batches a day.
		
		\item Storage of LPG is done as follows:-
		\begin{itemize}
			\item 3 Horton Spheres 1400 MT,1200 X 2 MT
			\item 1 Stationary Vessel ie. Bullet 150 X 4 MT
		\end{itemize}
		Therefore, net storage capacity = $1400+12 X 1200+150X4 = 4400 MT$
		\item Decantation is done via pressure difference using a vapour compressor in the following steps:-
		\begin{itemize}
			\item First vapour of TT is pressurized by taking vapour from vessel. This forces liquid LPG to move from TT to vessel.
			\item Then, liquid valve is closed and then vapour is sucked from the tanker using vapour suction.
		\end{itemize}
		This method is preferred over simply using pumps to pump the LPG from TT to vessel because if the pump pulls vapor by mistake, that will lead to cavitation.
		\item There are 3 carousels - 1400 cylinders/hour X 2 and 1600 cylinders per hour.
		\item The 3 carousels are fed by 3 pumps - 110 , 90, 36 X 2 CFM each with a max capacity of 6000 cylinders per month therefore, the net capacity would be approximately 18000 per month, but generally only about 15000 are required to be produced as per guidance from SO.
		\item The following requirement from SO side is generated by a computer model which takes the following factors into account -
		\begin{itemize}
			\item Bulk receiving cost
			\item Capacity of plant
			\item Demand from market
			\item Transportation cost from plant to market
		\end{itemize}
		\item There are 2 types of valves in cylinders ie. Self Closing Valve (SC) \ref{sc_valve}and Liquid Off Take Valves (LOT) \ref{lot_valve}.
		\item Delivery within 24 hours to distributor.
		\item There are baffle plates in LPG TTs to arrest momentum of the fluid thereby causing less hindrance to the driver.
	\end{itemize}
	\begin{wrapfigure}{o}{0.5\textwidth}
		\centering
		\includegraphics{lot_valve}
		\caption{LOT Valve}
		\label{lot_valve}
	\end{wrapfigure}
	\begin{wrapfigure}{o}{0.5\textwidth}
		\centering
		\includegraphics{sc_valve}
		\caption{SC Valve}
		\label{sc_valve}
	\end{wrapfigure}
	\section{FAQs}
	
\end{document}